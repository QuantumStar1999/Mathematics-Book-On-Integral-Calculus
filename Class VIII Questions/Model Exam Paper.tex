\documentclass[11pt,a4paper]{exam}
\usepackage{amsmath,amsfonts,amssymb,graphicx}
\usepackage[margin=1in]{geometry}
\usepackage{titlesec}
\usepackage{multicol}
\usepackage{datetime}

% Header and Footer
% Macros
\newcommand{\CONTENT}{Class VIII Mathematics}
\newcommand{\EXAMNAME}{Term I Examination}
\newcommand{\FULLMARKS}{Maximum Marks: 50}
\newcommand{\FULLTIME}{Time: 1 Hour 30 Minutes}
\pagestyle{headandfoot}
\firstpageheader{\textbf{\CONTENT}}{\textbf{\EXAMNAME}}{\textbf{\FULLMARKS}}
\runningheader{ \CONTENT}{}{\textbf{\EXAMNAME}}
\firstpagefooter{}{Page \thepage\ of \numpages}{}
\runningfooter{}{Page \thepage\ of \numpages}{}

% Section formatting
\titleformat{\section}{\large\bfseries\filcenter}{}{0em}{}
% Macros for section
\newcommand{\GetInstructions}{
	\section*{General Instructions:}
	\begin{enumerate}
		\item All questions are compulsory
		\item Show all working steps for full marks
		\item Use mathematical reasoning to justify your answers
		\item Calculators are not permitted
		\item Extra credit will be given for elegant solutions
	\end{enumerate}
}
\begin{document}
	
	\begin{center}
		\Large\textbf{GUIDELY MATHEMATICS CHALLENGE} \\
		\large\textbf{Class VIII} \\
		\textbf{Topics: Rational Numbers, Square Numbers, Cube Numbers} \\
		\textbf{\FULLTIME \hspace{1cm} \FULLMARKS} \\
		\rule{\textwidth}{1pt}
	\end{center}
	
	%\GetInstructions
	
	\section*{Section A - Conceptual Understanding (1 mark each)}
	\begin{questions}
		\question If $x = \frac{\sqrt{7} - \sqrt{5}}{\sqrt{7} + \sqrt{5}}$ and $y = \frac{\sqrt{7} + \sqrt{5}}{\sqrt{7} - \sqrt{5}}$, then $x + y$ is:
		\begin{choices}
			\choice $6$
			\choice $12$
			\choice $24$
			\choice $2\sqrt{35}$
		\end{choices}
		
		\question The cube of $\left(\frac{2}{3}\right)^{-2}$ is:
		\begin{choices}
			\choice $\frac{8}{27}$
			\choice $\frac{27}{8}$
			\choice $\frac{729}{64}$
			\choice $\frac{64}{729}$
		\end{choices}
		
		\question Which of these is always a perfect square for any integer $n$?
		\begin{choices}
			\choice $n^3 + n^2$
			\choice $(n+1)^3 - n^3$
			\choice $n^4 + 2n^2 + 1$
			\choice $n^5 - n$
		\end{choices}
		
		\question The rational number between $\frac{1}{3}$ and $\frac{1}{2}$ is:
		\begin{choices}
			\choice $\frac{7}{12}$
			\choice $\frac{5}{12}$
			\choice $\frac{9}{16}$
			\choice $\frac{11}{24}$
		\end{choices}
		
		\question If $n^3$ ends with 8, then $n^2$ must end with:
		\begin{choices}
			\choice $2$
			\choice $4$
			\choice $6$
			\choice $8$
		\end{choices}
		
		\question The smallest number by which 2592 must be multiplied to make it a perfect cube is:
		\begin{choices}
			\choice $2$
			\choice $3$
			\choice $6$
			\choice $12$
		\end{choices}
	\end{questions}
	
	\section*{Section B - Problem Solving (2 marks each)}
	\begin{questions}
		\question Find three rational numbers between $\frac{2}{7}$ and $\frac{3}{8}$ without taking average
		
		\question Prove that the cube of any odd number is always odd
		
		\question Simplify: $\left(\frac{125}{64}\right)^{-2/3} + \left(\frac{256}{625}\right)^{-1/4} + \left(\frac{\sqrt[3]{216}}{49}\right)^{-1/2}$
		
		\question If $\sqrt[3]{x} + \frac{1}{\sqrt[3]{x}} = 3$, find $x + \frac{1}{x}$
	\end{questions}
	
	\section*{Section C - Advanced Applications (3 marks each)}
	\begin{questions}
		\question Show that $0.3\overline{7}$ can be expressed as $\frac{17}{45}$, then generalize the method for any decimal of form $0.a\overline{b}$
		
		\question Find all perfect cubes between 1000 and 2000 that are also perfect squares
		
		\question Prove that $\sqrt[3]{2} + \sqrt[3]{4}$ is irrational using the fact that $\sqrt[3]{2}$ is irrational
		
		\question If $a$ and $b$ are positive rational numbers with $a \neq b$, prove that $\frac{a+b}{2} > \sqrt{ab}$ (AM-GM inequality for two numbers)
		
		\question Find all integer solutions to $x^3 - y^3 = 19$
	\end{questions}
	
	\section*{Section D - Proofs and Extended Problems (5 marks each)}
	\begin{questions}
		\question \textbf{(a)} Prove that the sum of a rational and irrational number is irrational \\
		\textbf{(b)} Hence prove that $\sqrt[3]{9} + \sqrt{3}$ is irrational
		
		\question \textbf{(a)} Find the smallest perfect square divisible by 12, 18, and 27 \\
		\textbf{(b)} Generalize the method for finding the smallest perfect square divisible by given numbers
		
		\question \textbf{(a)} If $x + \frac{1}{x} = 5$, find $x^3 + \frac{1}{x^3}$ \\
		\textbf{(b)} Develop a general formula for $x^n + \frac{1}{x^n}$ in terms of $x + \frac{1}{x}$
	\end{questions}
	
	\begin{center}
		\textbf{--- End of Advanced Challenge Paper ---}
	\end{center}
	
\end{document}