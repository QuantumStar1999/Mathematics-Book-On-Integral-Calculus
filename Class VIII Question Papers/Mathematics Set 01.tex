\documentclass[12pt]{article}
\usepackage{amsmath, amssymb, amsthm} % For mathematical symbols and equations
\usepackage{enumitem} % For customizing lists
\usepackage{geometry} % For page margins

\geometry{a4paper, margin=1in}

\title{Advanced Mathematics Question Paper}
\author{Class: VIII}
\date{Time: 1 Hour 30 Minutes \\ Max Marks: 40}

\begin{document}
	
	\maketitle
	
	\section*{Topic: Rational Numbers, Powers \& Exponents, Square Numbers}
	
	\subsection*{Section A: Challenging Multiple Choice Questions (1 Mark Each)}
	\textbf{Total: 5 $\times$ 1 = 5 Marks}
	
	\begin{enumerate}[label=\arabic*.]
		\item If $\left( \frac{a}{b} \right)^{x-2} = \left( \frac{b}{a} \right)^{x-4}$, then the value of $x$ is:
		\begin{enumerate}[label=\alph*)]
			\item 0
			\item 1
			\item 2
			\item 3
		\end{enumerate}
		
		\item The value of $\left( \frac{1}{1 + \sqrt{2}} + \frac{1}{\sqrt{2} + \sqrt{3}} + \frac{1}{\sqrt{3} + \sqrt{4}} \right)$ is:
		\begin{enumerate}[label=\alph*)]
			\item 0
			\item 1
			\item $\sqrt{4} - 1$
			\item $1 - \sqrt{4}$
		\end{enumerate}
		
		\item If $n$ is a positive integer, which of the following cannot be a perfect square?
		\begin{enumerate}[label=\alph*)]
			\item $5n^2$
			\item $9n^2$
			\item $16n^2$
			\item $20n^2$
		\end{enumerate}
		
		\item The rationalizing factor of $\frac{1}{\sqrt[3]{5} - \sqrt[3]{2}}$ is:
		\begin{enumerate}[label=\alph*)]
			\item $\sqrt[3]{5} + \sqrt[3]{2}$
			\item $\sqrt[3]{25} + \sqrt[3]{10} + \sqrt[3]{4}$
			\item $\sqrt[3]{5} - \sqrt[3]{2}$
			\item $\sqrt[3]{25} - \sqrt[3]{10} + \sqrt[3]{4}$
		\end{enumerate}
		
		\item If $x = \frac{\sqrt{3} + \sqrt{2}}{\sqrt{3} - \sqrt{2}}$ and $y = \frac{\sqrt{3} - \sqrt{2}}{\sqrt{3} + \sqrt{2}}$, then $x^2 + y^2$ is:
		\begin{enumerate}[label=\alph*)]
			\item 10
			\item 49
			\item 98
			\item 100
		\end{enumerate}
	\end{enumerate}
	
	\subsection*{Section B: Short Answer Questions (2 Marks Each)}
	\textbf{Total: 5 $\times$ 2 = 10 Marks}
	
	\begin{enumerate}[start=6]
		\item Find the value of $\left( \frac{64}{125} \right)^{-2/3} + \left( \frac{256}{625} \right)^{-1/4}$.
		
		\item If $\frac{5+2\sqrt{3}}{7+4\sqrt{3}} = a + b\sqrt{3}$, find the values of $a$ and $b$.
		
		\item Prove that $\left( \frac{x^a}{x^b} \right)^{a+b} \times \left( \frac{x^b}{x^c} \right)^{b+c} \times \left( \frac{x^c}{x^a} \right)^{c+a} = 1$.
		
		\item Find the smallest rationalizing factor of $\sqrt[4]{8}$.
		
		\item If $a = 2 + \sqrt{3}$, find $a^2 + \frac{1}{a^2}$.
	\end{enumerate}
	
	\subsection*{Section C: Long Answer Questions (3 Marks Each)}
	\textbf{Total: 5 $\times$ 3 = 15 Marks}
	
	\begin{enumerate}[start=11]
		\item Solve for $x$:
		\[ 2^{2x} - 6 \times 2^x + 8 = 0 \]
		
		\item If $x = \frac{\sqrt{5} + \sqrt{3}}{\sqrt{5} - \sqrt{3}}$ and $y = \frac{\sqrt{5} - \sqrt{3}}{\sqrt{5} + \sqrt{3}}$, find $x^2 + xy + y^2$.
		
		\item Prove that $\sqrt{5}$ is irrational by contradiction.
		
		\item Find the value of $\sqrt{6 + \sqrt{6 + \sqrt{6 + \dots}}}$.
		
		\item If $a = 9 + 4\sqrt{5}$, find $\sqrt{a} - \frac{1}{\sqrt{a}}$.
	\end{enumerate}
	
	\subsection*{Section D: Application-Based Problem (5 Marks)}
	\textbf{Total: 1 $\times$ 5 = 5 Marks}
	
	\begin{enumerate}[start=16]
		\item A right-angled triangle has legs of lengths $\sqrt{2}$ and $\sqrt{3}$.
		\begin{enumerate}[label=(\alph*)]
			\item Find the exact length of the hypotenuse.
			\item If a square has the same area as this triangle, find the side length of the square.
			\item Compare the perimeter of the triangle and the square.
		\end{enumerate}
	\end{enumerate}
	
	\section*{Marking Scheme (Sample Answers)}
	
	\subsection*{Section A}
	\begin{enumerate}
		\item d) 3 \quad (Hint: Equate exponents after expressing both sides with the same base.)
		\item c) $\sqrt{4} - 1$ \quad (Rationalize each term and observe telescoping cancellation.)
		\item d) 20n² \quad (20 is not a perfect square, so $20n^2$ can't be a perfect square.)
		\item b) $\sqrt[3]{25} + \sqrt[3]{10} + \sqrt[3]{4}$ \quad (Use $a^3 - b^3 = (a - b)(a^2 + ab + b^2)$.)
		\item c) 98 \quad (Compute $x + y = 10$ and $xy = 1$, then use $x^2 + y^2 = (x + y)^2 - 2xy$.)
	\end{enumerate}
	
	\subsection*{Section B}
	\begin{enumerate}[start=6]
		\item $\frac{25}{16} + \frac{5}{4} = \frac{45}{16}$ \quad (Simplify exponents and evaluate.)
		\item $a = 11, b = -6$ \quad (Rationalize the denominator and compare terms.)
		\item \textbf{Proof:} Combine exponents and simplify to 1.
		\item $\sqrt[4]{2}$ \quad (Multiply by $\sqrt[4]{8} \times \sqrt[4]{2} = \sqrt[4]{16} = 2$.)
		\item 14 \quad (Compute $a^2 = 7 + 4\sqrt{3}$ and $\frac{1}{a^2} = 7 - 4\sqrt{3}$.)
	\end{enumerate}
	
	\subsection*{Section C}
	\begin{enumerate}[start=11]
		\item $x = 1, 2$ \quad (Let $2^x = t$, solve quadratic $t^2 - 6t + 8 = 0$.)
		\item 49 \quad (Compute $x + y = 8$ and $xy = 1$, then $x^2 + xy + y^2 = (x + y)^2 - xy$.)
		\item \textbf{Proof by contradiction:} Assume $\sqrt{5} = \frac{p}{q}$ in lowest terms, derive contradiction.
		\item 3 \quad (Let $x = \sqrt{6 + x}$, solve $x^2 - x - 6 = 0$.)
		\item 4 \quad (Let $\sqrt{a} = \sqrt{5} + 2$, then compute $\sqrt{a} - \frac{1}{\sqrt{a}}$.)
	\end{enumerate}
	
	\subsection*{Section D}
	\begin{enumerate}[start=16]
		\item 
		\begin{enumerate}[label=(\alph*)]
			\item Hypotenuse = $\sqrt{ (\sqrt{2})^2 + (\sqrt{3})^2 } = \sqrt{5}$.
			\item Area of triangle = $\frac{1}{2} \times \sqrt{2} \times \sqrt{3} = \frac{\sqrt{6}}{2}$. Side of square = $\sqrt{ \frac{\sqrt{6}}{2} }$.
			\item Perimeter of triangle = $\sqrt{2} + \sqrt{3} + \sqrt{5}$. Perimeter of square = $4 \times \sqrt{ \frac{\sqrt{6}}{2} }$.
		\end{enumerate}
	\end{enumerate}
	
\end{document}